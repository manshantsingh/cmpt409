\documentclass{article}
\setlength\parindent{0pt}
\usepackage[hmargin=1in,vmargin=1in]{geometry}
\setlength{\parskip}{.5em}

\title{CMPT 409 Assignment 3}
\author{Heather Li, Ekjot Singh Billing, Manshant Singh Kohli}

\begin{document}
\maketitle

\section{Unidirectional TSP}
(C++11) 
For this question, we used a DP approach where we used two arrays $dp$ and $grid$ to store information about the minimal path starting from the left side of the matrix and terminating at a point $(i,j)$. 
\par
For each index $(i,j)$, $dp[i][j]$ stored the location of the penultimate step of the minimal path ending at location $(i,j)$ in the array. If there were multiple minimum paths ending at the same point, $dp[i][j]$ would store the smallest index.
\par 
Our array $grid$ stored the weight of the minimum path ending on $(i,j)$. In $O(n^2)$ time, we iterated through the entire array (checking every row in every column from left to right).
\[grid[i][j]=\min(dp[i-1][j-1],dp[i][j-1],dp[i+1][i-1])+cost[i][j]\]
(where $cost[i][j]$ is the value stored in the matrix at entry $(i,j)$ and the row indices are taken modulo $m$).
\par 
Scanning our $grid$ array gives us the weight of the minimum cost path and the $dp$ array gives us the indices along its path.





\section{Chopsticks}
(C++ 11) For this question, we noted that to minimize badness in a set of three chopsticks A,B,C the chopsticks A and B must be consecutive elements in the sequence of lengths given. We stored an array $badness$ of the badness of consecutive elements.
\par
We created a dynamic programming array $dp$.
\[dp[i][j]=min(dp[i][j-1],dp[i-1][j-2]+badness[n-(j-2)] \]
$dp[i][j]$ stores the minimum badness that must be created if there must be $i$ sets of chopsticks, using the last $j$ lengths given to us in our input length. It considers $dp[i][j-1]$, the value if the $j-1$th last chopsticks were not included in a set and $dp[i-1][j-2]+badness[n-(j-2)]$, the badness if the $j-1$ and $j-2$ chopsticks were to be included in a set.
\par
Using such a DP function, we could recurse through the values in the array. If $3i<j$, then there wouldn't be enough space to fully create all the chopstick sets. If $i=0$, then no chopsticks sets needed to be created. Calling $dp[k+8][n]$ once our array is fully populated would return our solution.

\section{Chopsticks}
(C++ 11) For this question, we noted that if assigned directed edges between two countries which had a domination relationship, we could create a forest. Furthermore, by assigning a dummy root node, we could create a tree. We used a map to map the country names to indices and stored the domination/adjacency relationships in an adjacency list.
\par 
We established a dynamic programming array $dp$, where $dp[i][j]$ stored the minimum number of diamonds to bribe at least $j$ of country $i$ and its subjugates.
\par 
We used a recursive depth-first search to help us load our dp array. Our recursive function, when called on a country $"root"$, returned the number of countries under its (in)direct domination, including itself. We then iterated through each dominated country of $root$ and found the minimum amount of diamonds we could use if we wanted to gain $k$ of the dominated country's votes and $j-k$ of the dominating country's other votes.
\par 
Calling $dp[0][m]$ would give us the best way to get $m$ of the root country's (the dummy node which dominates all countries) votes.

\section{Distinct Subsequences}
(Java) First, we used a hash table to store the lists of the indices where every character in our subsequence $Z$ is stored in the main string $X$.
\par 
We then created a dp array $total$, where $total[i]$ contains the number of distinct subsequences of the first $i+1$ characters found in $X$. We populated this dynamic programming array by adding  $total[i-1]$ to $total[i]$ for every occurence of a character of $Z$ at index $i$.
\par 
Once our dp array $total$ is populated, the answer is found at its last entry.


\end{document}